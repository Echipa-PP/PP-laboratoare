%%%%%%%%%%%%%%%%%%%%%%%%%%%%%%%%%%%%%%%%%%%%%%%%%%%%%%%
% MatPlotLib and Random Cheat Sheet
%
% Edited by Michelle Cristina de Sousa Baltazar
%
% http://matplotlib.org/api/pyplot_summary.html
% http://matplotlib.org/users/pyplot_tutorial.html
%
%%%%%%%%%%%%%%%%%%%%%%%%%%%%%%%%%%%%%%%%%%%%%%%%%%%%%%%

\documentclass[a3paper]{article}
\usepackage[landscape]{geometry}
\usepackage{url}
\usepackage{multicol}
\usepackage{amsmath}
\usepackage{amsfonts}
\usepackage{tikz}
\usetikzlibrary{decorations.pathmorphing}
\usepackage{amsmath,amssymb}
\usepackage{hyperref}

\usepackage{colortbl}
\usepackage{xcolor}
\usepackage{mathtools}
\usepackage{amsmath,amssymb}
\usepackage{enumitem}

% Mihnea
\usepackage{textcomp} % \textquotesingle: Racket: '(1 2 3)
\usepackage{couriers}
\usepackage{listings}
\lstset{
	numbers			= left,
	numberstyle		= \tiny,
    numbersep       = 5pt,
	captionpos		= b,
	breaklines		= true,
	basicstyle		= \ttfamily\footnotesize, 
	tabsize			= 4,
	escapeinside	= {~}{~},
}
\lstdefinelanguage{Racket}{
  morekeywords=[1]{define, define-syntax, define-macro, lambda, define-stream, stream-lambda},
  morekeywords=[2]{begin, call-with-current-continuation, call/cc,
    call-with-input-file, call-with-output-file, case, cond,
    do, else, for-each, if,
    let*, let, let-syntax, letrec, letrec-syntax,
    let-values, let*-values,
    and, or, not, delay, force,
    quasiquote, quote, unquote, unquote-splicing,
    map, fold, syntax, syntax-rules, eval, environment },
  morekeywords=[3]{import, export},
  alsodigit=!\$\%&*+-./:<=>?@^_~,
  sensitive=true,
  morecomment=[l]{;},
  morecomment=[s]{\#|}{|\#},
  morestring=[b]",
  basicstyle=\footnotesize\ttfamily,
  keywordstyle=\color[rgb]{0,.3,.7},
  commentstyle=\color[rgb]{0.133,0.545,0.133},
  stringstyle={\color[rgb]{0.75,0.49,0.07}},
  upquote=true,
  breaklines=true,
  breakatwhitespace=true,
  literate=*{`}{{`}}{1}
}

\title{Prolog Cheatsheet}
\usepackage[brazilian]{babel}
\usepackage[utf8]{inputenc}

\advance\topmargin-1.0in
\advance\textheight3in
\advance\textwidth3in
\advance\oddsidemargin-1.5in
\advance\evensidemargin-1.5in
\parindent0pt
\parskip1pt
\newcommand{\hr}{\centerline{\rule{3.5in}{1pt}}}
%\colorbox[HTML]{e4e4e4}{\makebox[\textwidth-2\fboxsep][l]{texto}
\begin{document}

\begin{center}{\huge{\textbf{Prolog CheatSheet}}}\\
{\large Laborator 11}
\end{center}

\begin{multicols*}{3}

\tikzstyle{mybox} = [draw=black, fill=white, very thick,
    rectangle, rounded corners, inner sep=10pt, inner ysep=10pt]
\tikzstyle{fancytitle} =[fill=black, text=white, font=\bfseries]

% Mihnea
\tikzstyle{mybox_code} = [mybox, draw = orange, fill=sandybrown]
\tikzstyle{fancytitle_code} = [fancytitle, fill = orange]

\definecolor{almond}{rgb}{0.94, 0.87, 0.8}
\definecolor{apricot}{rgb}{0.98, 0.81, 0.69}
\definecolor{atomictangerine}{rgb}{1.0, 0.6, 0.4}
\definecolor{sandybrown}{rgb}{0.96, 0.64, 0.38}
\definecolor{buff}{rgb}{0.94, 0.86, 0.51}

\definecolor{persianred}{rgb}{0.8, 0.2, 0.2}
\definecolor{papayawhip}{rgb}{1.0, 0.94, 0.84}
\tikzstyle{mybox_persianred} = [mybox, draw = persianred, fill=papayawhip]
\tikzstyle{fancytitle_persianred} = [fancytitle, fill = persianred]

\definecolor{whitesmoke}{rgb}{0.96, 0.96, 0.96}
\definecolor{wenge}{rgb}{0.39, 0.33, 0.32}
\tikzstyle{mybox_blue} = [mybox, draw = wenge, fill=whitesmoke]
\tikzstyle{fancytitle_blue} = [fancytitle, fill = wenge]

\definecolor{cerise}{rgb}{0.87, 0.19, 0.39}
\definecolor{mistyrose}{rgb}{1.0, 0.89, 0.88}
\tikzstyle{mybox_cerise} = [mybox, draw = cerise, fill=mistyrose]
\tikzstyle{fancytitle_cerise} = [fancytitle, fill = cerise]

\definecolor{pinegreen}{rgb}{0.0, 0.47, 0.44}
\definecolor{bubbles}{rgb}{0.91, 1.0, 1.0}
\tikzstyle{mybox_pinegreen} = [mybox, draw = pinegreen, fill=bubbles]
\tikzstyle{fancytitle_pinegreen} = [fancytitle, fill = pinegreen]

\definecolor{cream}{rgb}{1.0, 0.99, 0.82}
\definecolor{mikadoyellow}{rgb}{1.0, 0.77, 0.05}
\tikzstyle{mybox_mikadoyellow} = [mybox, draw = mikadoyellow, fill=cream]
\tikzstyle{fancytitle_mikadoyellow} = [fancytitle, fill = mikadoyellow]

\definecolor{cornsilk}{rgb}{1.0, 0.97, 0.86}
\tikzstyle{mybox_orange} = [mybox, draw = orange, fill=cornsilk]
\tikzstyle{fancytitle_orange} = [fancytitle, fill = orange]

\definecolor{aliceblue}{rgb}{0.94, 0.97, 1.0}
\definecolor{seagreen}{rgb}{0.18, 0.55, 0.34}
\tikzstyle{mybox_seagreen} = [mybox, draw = seagreen, fill=aliceblue]
\tikzstyle{fancytitle_seagreen} = [fancytitle, fill = seagreen]

\definecolor{jazzberryjam}{rgb}{0.65, 0.04, 0.37}
\definecolor{almond}{rgb}{0.94, 0.87, 0.8}
\tikzstyle{mybox_jazzberryjam} = [mybox, draw = jazzberryjam, fill=almond]
\tikzstyle{fancytitle_jazzberryjam} = [fancytitle, fill = jazzberryjam]

\definecolor{amaranth}{rgb}{0.9, 0.17, 0.31}
\definecolor{bisque}{rgb}{1.0, 0.89, 0.77}
\tikzstyle{mybox_amaranth} = [mybox, draw = amaranth, fill=bisque]
\tikzstyle{fancytitle_amaranth} = [fancytitle, fill = amaranth]

\definecolor{carminered}{rgb}{1.0, 0.0, 0.22}
\definecolor{blanchedalmond}{rgb}{1.0, 0.92, 0.8}
\tikzstyle{mybox_carminered} = [mybox, draw = amaranth, fill=blanchedalmond]
\tikzstyle{fancytitle_carminered} = [fancytitle, fill = carminered]

\definecolor{midnightgreen}{rgb}{0.0, 0.29, 0.33}
\definecolor{lavendermist}{rgb}{0.9, 0.9, 0.98}
\tikzstyle{mybox_midnightgreen} = [mybox, draw = midnightgreen, fill=lavendermist]
\tikzstyle{fancytitle_midnightgreen} = [fancytitle, fill = midnightgreen]

\definecolor{indigo}{rgb}{0.29, 0.0, 0.51}
\definecolor{isabelline}{rgb}{0.96, 0.94, 0.93}
\tikzstyle{mybox_indigo} = [mybox, draw = indigo, fill=isabelline]
\tikzstyle{fancytitle_indigo} = [fancytitle, fill = indigo]

\definecolor{russet}{rgb}{0.5, 0.27, 0.11}
\definecolor{ivory}{rgb}{1.0, 1.0, 0.94}
\tikzstyle{mybox_russet} = [mybox, draw = russet, fill=ivory]
\tikzstyle{fancytitle_russet} = [fancytitle, fill = russet]

\definecolor{neongreen}{rgb}{0.12, 0.58, 0.02}
\definecolor{splashedwhite}{rgb}{0.9, 0.99, 0.9}
\tikzstyle{mybox_neongreen} = [mybox, draw = neongreen, fill=splashedwhite]
\tikzstyle{fancytitle_neongreen} = [fancytitle, fill = neongreen]

%---------------------------------------------------------------------------------
\begin{tikzpicture}
\node [mybox_persianred] (box){%
    \begin{minipage}{0.3\textwidth}
 
\begin{lstlisting}[language=Prolog, numbers=none]
[]                  - lista vida
[a,b,c]             - elementele a, b, c
[Prim|Rest]         - element concatenat la rest
[X1,X2,...,XN|Rest] - n elemente concatenate la restul listei

append(?List1, ?List2, ?List1AndList2) % ~List1AndList2 este concatenarea între List1 și List2~
?- append([1,2,3], [4,5], X). 
X = [1, 2, 3, 4, 5].
?- append([1,2,3], X, [1,2,3,4]).
X = [4].
?- append([1,2,3], X, [2,3,4]).
false.

member(?Elem, ?List) % ~Adevărat dacă Elem esre prezent în List~
?- member(1, [1,2,3]).
true ;
false.
?- member(X, [1,2,3]).
X = 1 ;
X = 2 ;
X = 3.

length(?List, ?Int) % ~Adevărat dacă Int reprezintă numărul de elemente din List~
?- length([1,2,3], X).
X = 3.

reverse(?List1, ?List2) % ~Adevărat atunci când elementele din List2 sunt în ordine inversă față de List1~
?- reverse([1, 2, 3], X).
X = [3, 2, 1].

sort(+List, -Sorted) % ~Adevărat atunci când Sorted conține toate elementele din List (fără duplicate), sortate în ordinea standard~
?- sort([2, 3, 1, 2], X).
X = [1, 2, 3].
\end{lstlisting}

\end{minipage}
};
\node[fancytitle_persianred, right=10pt] at (box.north west) {Liste și funcții utile pe liste};
\end{tikzpicture}

\begin{tikzpicture}
\node [mybox_pinegreen] (box){%
\begin{minipage}{0.3\textwidth}
\begin{small}
Regulă: Atunci când calea către soluţie respectă un anumit template (avem de instanţiat un număr finit, predeterminat, de variabile), este eficient să definim un astfel de template în program.
\end{small}

\begin{small}
De exemplu, pentru problema celor opt regine putem scrie astfel:
\end{small}
\begin{lstlisting}[language=Prolog, numbers=none]
template([1/_, 2/_, 3/_, 4/_, 5/_, 6/_, 7/_, 8/_]).

correct([]):-!.
correct([X/Y|Others]):-
        correct(Others),
        member(Y, [1, 2, 3, 4, 5, 6, 7, 8]),
        safe(X/Y, Others). % ~predicat care verifică faptul că reginele nu se atacă între ele~

solve_queens(S):-template(S), correct(S).
\end{lstlisting}
\end{minipage}
};
\node[fancytitle_pinegreen, right=10pt] at (box.north west) {Backtracking când se cunoaște lungimea căii către soluție};
\end{tikzpicture}


\begin{tikzpicture}
\node [mybox_orange] (box){%
    \begin{minipage}{0.3\textwidth}
\begin{small}
În această situaţie nu este posibil să definim un template care descrie forma soluţiei problemei. Vom defini o căutare mai generală, după modelul următor:
\end{small}
\begin{lstlisting}[language=Prolog, numbers=none]
solve(Solution):-
        initial_state(State),
        search([State], Solution).
\end{lstlisting}
\begin{small}
search(+StăriVizitate,-Soluţie) definește mecanismul general de căutare astfel:
\begin{itemize}
\item căutarea începe de la o stare inițială dată (predicatul initial\_state/1)
\item dintr-o stare curentă se generează stările următoare posibile (predicatul next\_state/2)
\item se testează că starea în care s-a trecut este nevizitată anterior (evitând astfel traseele ciclice)
\item căutarea continuă din noua stare, până se întâlneşte o stare finală (predicatul final\_state/1)
\end{itemize}
\end{small}
\begin{lstlisting}[language=Prolog, numbers=none]
search([CurrentState|Other], Solution):-
        final_state(CurrentState), !,
        reverse([CurrentState|Other], Solution).

search([CurrentState|Other], Solution):-
        next_state(CurrentState, NextState),
        \+ member(NextState, Other),
        search([NextState,CurrentState|Other], Solution).
\end{lstlisting}
\end{minipage}
};

\node[fancytitle_orange, right=10pt] at (box.north west) {Backtracking când calea are un număr nedeterminat de stări};
\end{tikzpicture}

\begin{tikzpicture}
\node [mybox_amaranth] (box){%
    \begin{minipage}{0.3\textwidth}

\begin{small}
bfs(+CoadaStărilorNevizitate,+StăriVizitate,-Soluţie) va defini mecanismul general de căutare în lăţime, astfel:
\begin{itemize}
\item căutarea începe de la o stare inițială dată care n-are predecesor în spaţiul stărilor (StartNode cu părintele nil)
\item se generează toate stările următoare posibile
\item se adaugă toate aceste stări la coada de stări încă nevizitate
\item căutarea continuă din starea aflată la începutul cozii, până se întâlneşte o stare finală
\end{itemize}
\end{small}
\begin{lstlisting}[language=Prolog, numbers=none]
bfs([(StartNode,nil)], [], Discovered).
\end{lstlisting}

    \end{minipage}
};

\node[fancytitle_amaranth, right=10pt] at (box.north west) {Mecanism BFS};
\end{tikzpicture}




\begin{tikzpicture}
\node [mybox_amaranth] (box){%
    \begin{minipage}{0.3\textwidth}
\begin{small}
astar(+End, +Frontier, +Discovered, +Grid, -Result) va defini mecanismul general de căutare A*, astfel:
\begin{itemize}
\item căutarea începe de la o stare inițială dată care n-are predecesor în spaţiul stărilor (StartNode cu părintele nil) și distanța estimată de la acesta până la nodul de final printr-o euristică (exemplu: distanța Manhattan)
\item se generează toate stările următoare posibile și se calculează costul acestora adăugând costul acțiunii din părinte până în starea aleasă cu costul real calculat pentru a ajunge în părinte (costul părintelui în Discovered)
\item dacă starea aleasă nu este în Discovered sau dacă noul cost calculat al acesteia este mai mic decât cel din Discovered, se adaugă în acesta, apoi va fi introdusă în coada de priorități (Frontier) cu prioritatea fiind costul cu care a fost adaugată în Discovered + valoarea dată de euristică din starea curentă până în cea finală
\item căutarea continuă din starea aflată la începutul cozii, până se întâlneşte o stare finală
\end{itemize}
\end{small}
\begin{lstlisting}[language=Prolog, numbers=none]
astar_search(Start, End, Grid, Path) :-
  manhattan(Start, End, H),
  astar(End, [H:Start], [Start:("None", 0)], Grid, Discovered),
  get_path(Start, End, Discovered, [End], Path).
\end{lstlisting}

    \end{minipage}
};

\node[fancytitle_amaranth, right=10pt] at (box.north west) {Mecanism A*};
\end{tikzpicture}






\end{multicols*}
\end{document}
Contact GitHub API Training Shop Blog About
© 2016 GitHub, Inc. Terms Privacy Security Status Help