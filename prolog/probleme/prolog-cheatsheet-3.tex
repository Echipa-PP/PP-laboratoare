%%%%%%%%%%%%%%%%%%%%%%%%%%%%%%%%%%%%%%%%%%%%%%%%%%%%%%%
% MatPlotLib and Random Cheat Sheet
%
% Edited by Michelle Cristina de Sousa Baltazar
%
% http://matplotlib.org/api/pyplot_summary.html
% http://matplotlib.org/users/pyplot_tutorial.html
%
%%%%%%%%%%%%%%%%%%%%%%%%%%%%%%%%%%%%%%%%%%%%%%%%%%%%%%%

\documentclass[a3paper]{article}
\usepackage[landscape]{geometry}
\usepackage{url}
\usepackage{multicol}
\usepackage{amsmath}
\usepackage{amsfonts}
\usepackage{tikz}
\usetikzlibrary{decorations.pathmorphing}
\usepackage{amsmath,amssymb}
\usepackage{hyperref}

\usepackage{colortbl}
\usepackage{xcolor}
\usepackage{mathtools}
\usepackage{amsmath,amssymb}
\usepackage{enumitem}

% Mihnea
\usepackage{textcomp} % \textquotesingle: Racket: '(1 2 3)
\usepackage{couriers}
\usepackage{listings}
\lstset{
	numbers			= left,
	numberstyle		= \tiny,
    numbersep       = 5pt,
	captionpos		= b,
	breaklines		= true,
	basicstyle		= \ttfamily\footnotesize, 
	tabsize			= 4,
	escapeinside	= {~}{~},
}
\lstdefinelanguage{Racket}{
  morekeywords=[1]{define, define-syntax, define-macro, lambda, define-stream, stream-lambda},
  morekeywords=[2]{begin, call-with-current-continuation, call/cc,
    call-with-input-file, call-with-output-file, case, cond,
    do, else, for-each, if,
    let*, let, let-syntax, letrec, letrec-syntax,
    let-values, let*-values,
    and, or, not, delay, force,
    quasiquote, quote, unquote, unquote-splicing,
    map, fold, syntax, syntax-rules, eval, environment },
  morekeywords=[3]{import, export},
  alsodigit=!\$\%&*+-./:<=>?@^_~,
  sensitive=true,
  morecomment=[l]{;},
  morecomment=[s]{\#|}{|\#},
  morestring=[b]",
  basicstyle=\footnotesize\ttfamily,
  keywordstyle=\color[rgb]{0,.3,.7},
  commentstyle=\color[rgb]{0.133,0.545,0.133},
  stringstyle={\color[rgb]{0.75,0.49,0.07}},
  upquote=true,
  breaklines=true,
  breakatwhitespace=true,
  literate=*{`}{{`}}{1}
}

\title{Prolog Cheatsheet}
\usepackage[brazilian]{babel}
\usepackage[utf8]{inputenc}

\advance\topmargin-1.0in
\advance\textheight3in
\advance\textwidth3in
\advance\oddsidemargin-1.5in
\advance\evensidemargin-1.5in
\parindent0pt
\parskip1pt
\newcommand{\hr}{\centerline{\rule{3.5in}{1pt}}}
%\colorbox[HTML]{e4e4e4}{\makebox[\textwidth-2\fboxsep][l]{texto}
\begin{document}

\begin{center}{\huge{\textbf{Prolog CheatSheet}}}\\
{\large Laborator 12}
\end{center}

\begin{multicols*}{3}

\tikzstyle{mybox} = [draw=black, fill=white, very thick,
    rectangle, rounded corners, inner sep=10pt, inner ysep=10pt]
\tikzstyle{fancytitle} =[fill=black, text=white, font=\bfseries]

% Mihnea
\tikzstyle{mybox_code} = [mybox, draw = orange, fill=sandybrown]
\tikzstyle{fancytitle_code} = [fancytitle, fill = orange]

\definecolor{almond}{rgb}{0.94, 0.87, 0.8}
\definecolor{apricot}{rgb}{0.98, 0.81, 0.69}
\definecolor{atomictangerine}{rgb}{1.0, 0.6, 0.4}
\definecolor{sandybrown}{rgb}{0.96, 0.64, 0.38}
\definecolor{buff}{rgb}{0.94, 0.86, 0.51}

\definecolor{persianred}{rgb}{0.8, 0.2, 0.2}
\definecolor{papayawhip}{rgb}{1.0, 0.94, 0.84}
\tikzstyle{mybox_persianred} = [mybox, draw = persianred, fill=papayawhip]
\tikzstyle{fancytitle_persianred} = [fancytitle, fill = persianred]

\definecolor{whitesmoke}{rgb}{0.96, 0.96, 0.96}
\definecolor{wenge}{rgb}{0.39, 0.33, 0.32}
\tikzstyle{mybox_blue} = [mybox, draw = wenge, fill=whitesmoke]
\tikzstyle{fancytitle_blue} = [fancytitle, fill = wenge]

\definecolor{cerise}{rgb}{0.87, 0.19, 0.39}
\definecolor{mistyrose}{rgb}{1.0, 0.89, 0.88}
\tikzstyle{mybox_cerise} = [mybox, draw = cerise, fill=mistyrose]
\tikzstyle{fancytitle_cerise} = [fancytitle, fill = cerise]

\definecolor{pinegreen}{rgb}{0.0, 0.47, 0.44}
\definecolor{bubbles}{rgb}{0.91, 1.0, 1.0}
\tikzstyle{mybox_pinegreen} = [mybox, draw = pinegreen, fill=bubbles]
\tikzstyle{fancytitle_pinegreen} = [fancytitle, fill = pinegreen]

\definecolor{cream}{rgb}{1.0, 0.99, 0.82}
\definecolor{mikadoyellow}{rgb}{1.0, 0.77, 0.05}
\tikzstyle{mybox_mikadoyellow} = [mybox, draw = mikadoyellow, fill=cream]
\tikzstyle{fancytitle_mikadoyellow} = [fancytitle, fill = mikadoyellow]

\definecolor{cornsilk}{rgb}{1.0, 0.97, 0.86}
\tikzstyle{mybox_orange} = [mybox, draw = orange, fill=cornsilk]
\tikzstyle{fancytitle_orange} = [fancytitle, fill = orange]

\definecolor{aliceblue}{rgb}{0.94, 0.97, 1.0}
\definecolor{seagreen}{rgb}{0.18, 0.55, 0.34}
\tikzstyle{mybox_seagreen} = [mybox, draw = seagreen, fill=aliceblue]
\tikzstyle{fancytitle_seagreen} = [fancytitle, fill = seagreen]

\definecolor{jazzberryjam}{rgb}{0.65, 0.04, 0.37}
\definecolor{almond}{rgb}{0.94, 0.87, 0.8}
\tikzstyle{mybox_jazzberryjam} = [mybox, draw = jazzberryjam, fill=almond]
\tikzstyle{fancytitle_jazzberryjam} = [fancytitle, fill = jazzberryjam]

\definecolor{amaranth}{rgb}{0.9, 0.17, 0.31}
\definecolor{bisque}{rgb}{1.0, 0.89, 0.77}
\tikzstyle{mybox_amaranth} = [mybox, draw = amaranth, fill=bisque]
\tikzstyle{fancytitle_amaranth} = [fancytitle, fill = amaranth]

\definecolor{carminered}{rgb}{1.0, 0.0, 0.22}
\definecolor{blanchedalmond}{rgb}{1.0, 0.92, 0.8}
\tikzstyle{mybox_carminered} = [mybox, draw = amaranth, fill=blanchedalmond]
\tikzstyle{fancytitle_carminered} = [fancytitle, fill = carminered]

\definecolor{midnightgreen}{rgb}{0.0, 0.29, 0.33}
\definecolor{lavendermist}{rgb}{0.9, 0.9, 0.98}
\tikzstyle{mybox_midnightgreen} = [mybox, draw = midnightgreen, fill=lavendermist]
\tikzstyle{fancytitle_midnightgreen} = [fancytitle, fill = midnightgreen]

\definecolor{indigo}{rgb}{0.29, 0.0, 0.51}
\definecolor{isabelline}{rgb}{0.96, 0.94, 0.93}
\tikzstyle{mybox_indigo} = [mybox, draw = indigo, fill=isabelline]
\tikzstyle{fancytitle_indigo} = [fancytitle, fill = indigo]

\definecolor{russet}{rgb}{0.5, 0.27, 0.11}
\definecolor{ivory}{rgb}{1.0, 1.0, 0.94}
\tikzstyle{mybox_russet} = [mybox, draw = russet, fill=ivory]
\tikzstyle{fancytitle_russet} = [fancytitle, fill = russet]

\definecolor{neongreen}{rgb}{0.12, 0.58, 0.02}
\definecolor{splashedwhite}{rgb}{0.9, 0.99, 0.9}
\tikzstyle{mybox_neongreen} = [mybox, draw = neongreen, fill=splashedwhite]
\tikzstyle{fancytitle_neongreen} = [fancytitle, fill = neongreen]

%---------------------------------------------------------------------------------


\begin{tikzpicture}
\node [mybox_seagreen] (box){%
    \begin{minipage}{0.3\textwidth}
    	{\centering\bf\small\color{seagreen} findall/3  \\}

\begin{lstlisting}[language=Prolog, numbers=none]
findall(+Template, +Goal, -Bag)
\end{lstlisting}
\begin{small}
Predicatul findall creează o listă de instanțieri ale lui Template care satisfac Goal și apoi unifică rezultatul cu Bag
\end{small}
\begin{lstlisting}[language=Prolog, numbers=none]
higherThan(Numbers, Element, Result):-
    findall(X, (member(X, Numbers), X > Element), Result).
?- higherThan([1, 2, 7, 9, 11], 5, X).
X = [7, 9, 11]

?- findall([X, SqX], (member(X, [1,2,7,9,15]), X > 5, SqX is X ** 2), Result). % ~în argumentul Template putem construi structuri mai complexe~
Result = [[7, 49], [9, 81], [15, 225]].

\end{lstlisting}
\end{minipage}
};

\node[fancytitle_seagreen, right=10pt] at (box.north west) {Aflarea tuturor soluțiilor pentru satisfacerea unui scop};
\end{tikzpicture}


\begin{tikzpicture}
\node [mybox_pinegreen] (box){%
\begin{minipage}{0.3\textwidth}
    	{\centering\bf\small\color{seagreen} bagof/3  \\}

\begin{lstlisting}[language=Prolog, numbers=none]
bagof(+Template, +Goal, -Bag)
\end{lstlisting}
\begin{small}
Predicatul bagof este asemănător cu predicatul findall, cu excepția faptului că predicatul bagof construiește câte o listă separată pentru fiecare instanțiere diferită a variabilelor din Goal (fie că ele sunt numite sau sunt înlocuite cu underscore.
\end{small}
\begin{lstlisting}[language=Prolog, numbers=none]
are(andrei,laptop,1). are(andrei,pix,5). are(andrei,ghiozdan,2).
are(radu,papagal,1). are(radu,ghiozdan,1). are(radu,laptop,2).
are(ana, telefon, 3). are(ana, masina, 1).

?- findall(X, are(_, X, _), Bag).
Bag = [laptop, pix, ghiozdan, papagal, ghiozdan, laptop, telefon, masina]. % ~laptop și ghiozdan apar de două ori pentru că sunt două posibile legări pentru persoană și pentru cantitate~

?- bagof(X, are(andrei, X, _), Bag).
Bag = [laptop] ;
Bag = [ghiozdan] ;
Bag = [pix].
% ~bagof creează câte o soluție pentru fiecare posibilă legare pentru cantitate. Putem aici folosi operatorul existențial ~^
?- bagof(X, C^are(andrei, X, C), Bag).
Bag = [laptop, pix, ghiozdan]. % ~am cerut lui bagof să pună toate soluțiile indiferent de legarea lui C în același grup~

?- bagof(X, C^are(P, X, C), Bag).
P = ana, Bag = [telefon, masina] ;
P = andrei, Bag = [laptop, pix, ghiozdan] ;
P = radu, Bag = [papagal, ghiozdan, laptop].
\end{lstlisting}

Dacă aplicăm operatorul existențial pe toate variabilele libere din scop, rezultatul este identic cu cel al lui \texttt{findall}.
\begin{lstlisting}[language=Prolog, numbers=none]
?- bagof(X, X^P^C^are(P, X, C), Bag).
Bag = [laptop, pix, ghiozdan, papagal, ghiozdan, laptop, telefon, masina].
\end{lstlisting}
\end{minipage}
};

\node[fancytitle_seagreen, right=10pt] at (box.north west) {Aflarea tuturor soluțiilor pentru satisfacerea unui scop};
\end{tikzpicture}


\begin{tikzpicture}
\node [mybox_pinegreen] (box){%
\begin{minipage}{0.3\textwidth}
    	{\centering\bf\small\color{seagreen} setof/3 \\}


\begin{lstlisting}[language=Prolog, numbers=none]
setof(+Template, +Goal, -Bag)
\end{lstlisting}
\begin{small}
Predicatul setof este asemănător cu bagof, dar sortează rezultatul (și elimină duplicatele) folosind sort/2.
\end{small}
\begin{lstlisting}[language=Prolog, numbers=none]
?- setof(X, C^are(P, X, C), Bag).
P = ana, Bag = [masina, telefon] ; %~se observă sortarea~
P = andrei, Bag = [ghiozdan, laptop, pix] ;
P = radu, Bag = [ghiozdan, laptop, papagal].

?- setof(X, P^C^are(P, X, C), Bag).%~ setof elimină duplicatele~
Bag = [ghiozdan, laptop, masina, papagal, pix, telefon]. 
\end{lstlisting}
\end{minipage}
};

\node[fancytitle_seagreen, right=10pt] at (box.north west) {Aflarea tuturor soluțiilor pentru satisfacerea unui scop};
\end{tikzpicture}


\end{multicols*}
\end{document}
Contact GitHub API Training Shop Blog About
© 2016 GitHub, Inc. Terms Privacy Security Status Help