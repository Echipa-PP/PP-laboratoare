%%%%%%%%%%%%%%%%%%%%%%%%%%%%%%%%%%%%%%%%%%%%%%%%%%%%%%%
% MatPlotLib and Random Cheat Sheet
%
% Edited by Michelle Cristina de Sousa Baltazar
%
% http://matplotlib.org/api/pyplot_summary.html
% http://matplotlib.org/users/pyplot_tutorial.html
%
%%%%%%%%%%%%%%%%%%%%%%%%%%%%%%%%%%%%%%%%%%%%%%%%%%%%%%%

\documentclass[a4paper]{article}
\usepackage[landscape]{geometry}
\usepackage{url}
\usepackage{multicol}
\usepackage{amsmath}
\usepackage{amsfonts}
\usepackage{tikz}
\usetikzlibrary{decorations.pathmorphing}
\usepackage{amsmath,amssymb}
\usepackage{hyperref}

\usepackage{colortbl}
\usepackage{xcolor}
\usepackage{mathtools}
\usepackage{amsmath,amssymb}
\usepackage{enumitem}

% Mihnea
\usepackage{textcomp} % \textquotesingle: Racket: '(1 2 3)
\usepackage{couriers}
\usepackage{listings}
\lstset{
	numbers			= left,
	numberstyle		= \tiny,
    numbersep       = 5pt,
	captionpos		= b,
	breaklines		= true,
	basicstyle		= \ttfamily\footnotesize, 
	tabsize			= 4,
	escapeinside	= {~}{~},
}
\lstdefinelanguage{Racket}{
  morekeywords=[1]{define, define-syntax, define-macro, lambda, define-stream, stream-lambda},
  morekeywords=[2]{begin, call-with-current-continuation, call/cc,
    call-with-input-file, call-with-output-file, case, cond,
    do, else, for-each, if,
    let*, let, let-syntax, letrec, letrec-syntax,
    let-values, let*-values,
    and, or, not, delay, force,
    quasiquote, quote, unquote, unquote-splicing,
    map, fold, syntax, syntax-rules, eval, environment },
  morekeywords=[3]{import, export},
  alsodigit=!\$\%&*+-./:<=>?@^_~,
  sensitive=true,
  morecomment=[l]{;},
  morecomment=[s]{\#|}{|\#},
  morestring=[b]",
  basicstyle=\footnotesize\ttfamily,
  keywordstyle=\color[rgb]{0,.3,.7},
  commentstyle=\color[rgb]{0.133,0.545,0.133},
  stringstyle={\color[rgb]{0.75,0.49,0.07}},
  upquote=true,
  breaklines=true,
  breakatwhitespace=true,
  literate=*{`}{{`}}{1}
}

\title{Racket - fluxuri}
\usepackage[brazilian]{babel}
\usepackage[utf8]{inputenc}

\advance\topmargin-1.0in
\advance\textheight3in
\advance\textwidth3in
\advance\oddsidemargin-1.5in
\advance\evensidemargin-1.5in
\parindent0pt
\parskip1pt
\newcommand{\hr}{\centerline{\rule{3.5in}{1pt}}}
%\colorbox[HTML]{e4e4e4}{\makebox[\textwidth-2\fboxsep][l]{texto}
\begin{document}

\begin{center}{\huge{\textbf{Racket CheatSheet}}}\\
{\large Laborator5}
\end{center}
\begin{multicols*}{3}

\tikzstyle{mybox} = [draw=black, fill=white, very thick,
    rectangle, rounded corners, inner sep=10pt, inner ysep=10pt]
\tikzstyle{fancytitle} =[fill=black, text=white, font=\bfseries]

% Mihnea
\tikzstyle{mybox_code} = [mybox, draw = orange, fill=sandybrown]
\tikzstyle{fancytitle_code} = [fancytitle, fill = orange]

\definecolor{almond}{rgb}{0.94, 0.87, 0.8}
\definecolor{apricot}{rgb}{0.98, 0.81, 0.69}
\definecolor{atomictangerine}{rgb}{1.0, 0.6, 0.4}
\definecolor{sandybrown}{rgb}{0.96, 0.64, 0.38}
\definecolor{buff}{rgb}{0.94, 0.86, 0.51}

\definecolor{persianred}{rgb}{0.8, 0.2, 0.2}
\definecolor{papayawhip}{rgb}{1.0, 0.94, 0.84}
\tikzstyle{mybox_persianred} = [mybox, draw = persianred, fill=papayawhip]
\tikzstyle{fancytitle_persianred} = [fancytitle, fill = persianred]

\definecolor{lavenderblue}{rgb}{0.9, 0.9, 1.0}
\definecolor{blue}{rgb}{0.01, 0.31, 0.59}
\tikzstyle{mybox_blue} = [mybox, draw = blue, fill=lavenderblue]
\tikzstyle{fancytitle_blue} = [fancytitle, fill = blue]

\definecolor{cerise}{rgb}{0.87, 0.19, 0.39}
\definecolor{mistyrose}{rgb}{1.0, 0.89, 0.88}
\tikzstyle{mybox_cerise} = [mybox, draw = cerise, fill=mistyrose]
\tikzstyle{fancytitle_cerise} = [fancytitle, fill = cerise]

\definecolor{pinegreen}{rgb}{0.0, 0.47, 0.44}
\definecolor{bubbles}{rgb}{0.91, 1.0, 1.0}
\tikzstyle{mybox_pinegreen} = [mybox, draw = pinegreen, fill=bubbles]
\tikzstyle{fancytitle_pinegreen} = [fancytitle, fill = pinegreen]

\definecolor{cream}{rgb}{1.0, 0.99, 0.82}
\definecolor{mikadoyellow}{rgb}{1.0, 0.77, 0.05}
\tikzstyle{mybox_mikadoyellow} = [mybox, draw = mikadoyellow, fill=cream]
\tikzstyle{fancytitle_mikadoyellow} = [fancytitle, fill = mikadoyellow]

\definecolor{cornsilk}{rgb}{1.0, 0.97, 0.86}
\tikzstyle{mybox_orange} = [mybox, draw = orange, fill=cornsilk]
\tikzstyle{fancytitle_orange} = [fancytitle, fill = orange]

\definecolor{aliceblue}{rgb}{0.94, 0.97, 1.0}
\definecolor{seagreen}{rgb}{0.18, 0.55, 0.34}
\tikzstyle{mybox_seagreen} = [mybox, draw = seagreen, fill=aliceblue]
\tikzstyle{fancytitle_seagreen} = [fancytitle, fill = seagreen]

\definecolor{jazzberryjam}{rgb}{0.65, 0.04, 0.37}
\definecolor{almond}{rgb}{0.94, 0.87, 0.8}
\tikzstyle{mybox_jazzberryjam} = [mybox, draw = jazzberryjam, fill=almond]
\tikzstyle{fancytitle_jazzberryjam} = [fancytitle, fill = jazzberryjam]

\definecolor{amaranth}{rgb}{0.9, 0.17, 0.31}
\definecolor{bisque}{rgb}{1.0, 0.89, 0.77}
\tikzstyle{mybox_amaranth} = [mybox, draw = amaranth, fill=bisque]
\tikzstyle{fancytitle_amaranth} = [fancytitle, fill = amaranth]

\definecolor{carminered}{rgb}{1.0, 0.0, 0.22}
\definecolor{blanchedalmond}{rgb}{1.0, 0.92, 0.8}
\tikzstyle{mybox_carminered} = [mybox, draw = amaranth, fill=blanchedalmond]
\tikzstyle{fancytitle_carminered} = [fancytitle, fill = carminered]

\definecolor{midnightgreen}{rgb}{0.0, 0.29, 0.33}
\definecolor{lavendermist}{rgb}{0.9, 0.9, 0.98}
\tikzstyle{mybox_midnightgreen} = [mybox, draw = midnightgreen, fill=lavendermist]
\tikzstyle{fancytitle_midnightgreen} = [fancytitle, fill = midnightgreen]

\definecolor{indigo}{rgb}{0.29, 0.0, 0.51}
\definecolor{isabelline}{rgb}{0.96, 0.94, 0.93}
\tikzstyle{mybox_indigo} = [mybox, draw = indigo, fill=isabelline]
\tikzstyle{fancytitle_indigo} = [fancytitle, fill = indigo]

\definecolor{russet}{rgb}{0.5, 0.27, 0.11}
\definecolor{ivory}{rgb}{1.0, 1.0, 0.94}
\tikzstyle{mybox_russet} = [mybox, draw = russet, fill=ivory]
\tikzstyle{fancytitle_russet} = [fancytitle, fill = russet]

\definecolor{neongreen}{rgb}{0.22, 0.88, 0.08}
\definecolor{splashedwhite}{rgb}{1.0, 0.99, 1.0}
\tikzstyle{mybox_neongreen} = [mybox, draw = neongreen, fill=splashedwhite]
\tikzstyle{fancytitle_neongreen} = [fancytitle, fill = neongreen]

\definecolor{amber}{rgb}{1.0, 0.75, 0.0}
\definecolor{mybeige}{rgb}{1.0, 1.0, 0.95}
\tikzstyle{mybox_amber} = [mybox, draw = amber, fill=mybeige]
\tikzstyle{fancytitle_amber} = [fancytitle, fill = amber]
%---------------------------------------------------------------------------------
\begin{tikzpicture}
\node [mybox_indigo] (box){%
    \begin{minipage}{0.3\textwidth}
    {\centering\bf\small\color{indigo} delay force\\}
		\begin{lstlisting}[language=Racket]
(define p (delay (+ 1 2)))
p                                   #<promise:p>

;; force forteaza evaluarea
(force p)                                      3

;; un force subsecvent ia rezultatul din cache
(force p)                                      3
\end{lstlisting}
    \end{minipage}
};

\node[fancytitle_indigo, right=10pt] at (box.north west) {Promisiuni};
\end{tikzpicture}

%---------------------------------------------------------------------------------

\begin{tikzpicture}
\node [mybox_amber] (box){%
    \begin{minipage}{0.3\textwidth}
    	{\centering\bf\small\color{amber} empty-stream stream-cons\\}
		\begin{lstlisting}[language=Racket]
empty-stream						   #<stream>
(stream-cons 1 empty-stream)           #<stream>

(define ones (stream-cons 1 ones))   fluxul de 1

;; fluxul numerelor naturale
(define naturals                         
  (let loop ((n 0))
     (stream-cons n (loop (add1 n))))) 
        \end{lstlisting}
    \end{minipage}
};

\node[fancytitle_amber, right=10pt] at (box.north west) {Constructori fluxuri};
\end{tikzpicture}

%---------------------------------------------------------------------------------


\begin{tikzpicture}
\node [mybox_pinegreen] (box){%
    \begin{minipage}{0.3\textwidth}
    	{\centering\bf\small\color{pinegreen} stream-first stream-rest stream-empty? \\}
		\begin{lstlisting}[language=Racket]
(stream-first naturals)                        0
(stream-rest (stream-cons 2 ones))   fluxul de 1          

(stream-empty? empty-stream)                  #t 
(stream-empty? ones)                          #f

		\end{lstlisting}
    \end{minipage}
};

\node[fancytitle_pinegreen, right=10pt] at (box.north west) {Operatori pe fluxuri};
\end{tikzpicture}
%---------------------------------------------------------------------------------

\begin{tikzpicture}
\node [mybox_cerise] (box){%
    \begin{minipage}{0.3\textwidth}
        	{\centering\bf\small\color{cerise} stream-map stream-filter \\}
		\begin{lstlisting}[language=Racket]
;; stream-map merge numai cu functii unare
(stream-map sqr naturals)       fluxul 0, 1, 4..

(stream-filter even? naturals)    fluxul nr pare
        \end{lstlisting}
    \end{minipage}
};

\node[fancytitle_cerise, right=10pt] at (box.north west) {Funcționale pe fluxuri};
\end{tikzpicture}

%---------------------------------------------------------------------------------

\begin{tikzpicture}
\node [mybox_blue] (box){%
    \begin{minipage}{0.3\textwidth}
    {\centering\bf\small\color{blue} Generator recursiv \color{black} cu oricâți parametri\\}
    {\centering\bf\small definit în mod uzual cu \color{blue} named let\\}
		\begin{lstlisting}[language=Racket]
;; fluxul puterilor lui 2
(define powers-of-2
  (let loop ((n 1))
    (stream-cons n (loop (* n 2)))))
    
;; fluxul Fibonacci
(define fibonacci
  (let loop ((n1 0) (n2 1))
    (stream-cons n1 (loop n2 (+ n1 n2)))))
    
;; fluxul 1/(n!) 
;; (cu care putem aproxima constanta lui Euler)
(define rev-factorials
  (let loop ((term 1) (n 1))
    (stream-cons term (loop (/ term n) (add1 n)))))
    
;; testare: stream-take este definita de noi
;; in laborator, nu exista in Racket

;; rezultat '(1 2 4 8 16 32 64 128 256 512)
(stream-take powers-of-2 10)

;; rezultat '(0 1 1 2 3 5 8 13 21 34)
(stream-take fibonacci 10)

;; rezultat 2.7182815255731922
(apply + 0.0 (stream-take rev-factorials 10))
        \end{lstlisting}
    \end{minipage}
};

\node[fancytitle_blue, right=10pt] at (box.north west) {Fluxuri definite explicit};
\end{tikzpicture}

%---------------------------------------------------------------------------------

\begin{tikzpicture}
\node [mybox_persianred] (box){%
    \begin{minipage}{0.3\textwidth}
    {\centering\bf\small Folosiți \color{persianred} interfața Racket \color{black} pentru fluxuri!\\}
    \centering
%     	{\centering\bf\color{persianred}
		\begin{lstlisting}[language=Racket]
DA: (stream-cons x S) NU: (cons x (lambda () S))
                      NU: (cons x (delay S))
DA: (stream-rest S)   NU: ((cdr S))
                      NU: (force (cdr S))
        \end{lstlisting}
    \end{minipage}
};

\node[fancytitle_persianred, right=10pt] at (box.north west) {AȘA  DA / AȘA NU};
\end{tikzpicture}

%---------------------------------------------------------------------------------
\begin{tikzpicture}
\node [mybox_orange] (box){%
    \begin{minipage}{0.3\textwidth}
    {\centering\bf\small\color{orange} Fără generator explicit\\}
    \medskip
    {\centering\bf\small\color{orange} Dă explicit primii 1-2 termeni\color{black}, apoi inițiază\\} 
    {\centering\bf\small o prelucrare folosind (de obicei)\\} 
    {\centering\bf\small\color{orange}funcționale pe fluxuri\\}
    \begin{lstlisting}[language=Racket]
;; stream-zip-with este definita de voi
;; in laborator, nu exista in Racket

;; fluxul puterilor lui 2    
(define powers-of-2-a
  (stream-cons
   1
   (stream-zip-with +
                    powers-of-2-a
                    powers-of-2-a)))
                    
(define powers-of-2-b
  (stream-cons
   1
   (stream-map (lambda (x) (* x 2))
               powers-of-2-b)))                    
 
;; fluxul Fibonacci
(define fibonacci
  (stream-cons
   0
   (stream-cons
    1
    (stream-zip-with +
                     fibonacci
                     (stream-rest fibonacci)))))
    
;; fluxul 1/(n!) 
(define rev-factorials
  (stream-cons
   1
   (stream-zip-with /
                    rev-factorials
                    (stream-rest naturals))))
                    

        \end{lstlisting}
    \end{minipage}
};
\node[fancytitle_orange, right=10pt] at (box.north west) {Fluxuri definite implicit};
\end{tikzpicture}

%---------------------------------------------------------------------------------
\begin{tikzpicture}
\node [mybox_neongreen] (box){%
    \begin{minipage}{0.3\textwidth}\centering
\href{http://docs.racket-lang.org/}{http://docs.racket-lang.org/}
    \end{minipage}
};

\node[fancytitle_neongreen, right=10pt] at (box.north west) {Folositi cu incredere!};
\end{tikzpicture}



%---------------------------------------------------------------------------------



\end{multicols*}
\end{document}
