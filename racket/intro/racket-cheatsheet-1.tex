%%%%%%%%%%%%%%%%%%%%%%%%%%%%%%%%%%%%%%%%%%%%%%%%%%%%%%%
% MatPlotLib and Random Cheat Sheet
%
% Edited by Michelle Cristina de Sousa Baltazar
%
% http://matplotlib.org/api/pyplot_summary.html
% http://matplotlib.org/users/pyplot_tutorial.html
%
%%%%%%%%%%%%%%%%%%%%%%%%%%%%%%%%%%%%%%%%%%%%%%%%%%%%%%%

\documentclass[a4paper]{article}
\usepackage[landscape]{geometry}
\usepackage{url}
\usepackage{multicol}
\usepackage{amsmath}
\usepackage{amsfonts}
\usepackage{tikz}
\usetikzlibrary{decorations.pathmorphing}
\usepackage{amsmath,amssymb}
\usepackage{hyperref}

\usepackage{colortbl}
\usepackage{xcolor}
\usepackage{mathtools}
\usepackage{amsmath,amssymb}
\usepackage{enumitem}

% Mihnea
\usepackage{textcomp} % \textquotesingle: Racket: '(1 2 3)
\usepackage{couriers}
\usepackage{listings}
\lstset{
	numbers			= left,
	numberstyle		= \tiny,
    numbersep       = 5pt,
	captionpos		= b,
	breaklines		= true,
	basicstyle		= \ttfamily\footnotesize, 
	tabsize			= 4,
	escapeinside	= {~}{~},
}
\lstdefinelanguage{Racket}{
  morekeywords=[1]{define, define-syntax, define-macro, lambda, define-stream, stream-lambda},
  morekeywords=[2]{begin, call-with-current-continuation, call/cc,
    call-with-input-file, call-with-output-file, case, cond,
    do, else, for-each, if,
    let*, let, let-syntax, letrec, letrec-syntax,
    let-values, let*-values,
    and, or, not, delay, force,
    quasiquote, quote, unquote, unquote-splicing,
    map, fold, syntax, syntax-rules, eval, environment },
  morekeywords=[3]{import, export},
  alsodigit=!\$\%&*+-./:<=>?@^_~,
  sensitive=true,
  morecomment=[l]{;},
  morecomment=[s]{\#|}{|\#},
  morestring=[b]",
  basicstyle=\footnotesize\ttfamily,
  keywordstyle=\color[rgb]{0,.3,.7},
  commentstyle=\color[rgb]{0.133,0.545,0.133},
  stringstyle={\color[rgb]{0.75,0.49,0.07}},
  upquote=true,
  breaklines=true,
  breakatwhitespace=true,
  literate=*{`}{{`}}{1}
}

\title{Racket}
\usepackage[brazilian]{babel}
\usepackage[utf8]{inputenc}

\advance\topmargin-1.0in
\advance\textheight3in
\advance\textwidth3in
\advance\oddsidemargin-1.5in
\advance\evensidemargin-1.5in
\parindent0pt
\parskip1pt
\newcommand{\hr}{\centerline{\rule{3.5in}{1pt}}}
%\colorbox[HTML]{e4e4e4}{\makebox[\textwidth-2\fboxsep][l]{texto}
\begin{document}

\begin{center}{\huge{\textbf{Racket CheatSheet}}}\\
{\large Laborator1}
\end{center}
\begin{multicols*}{3}

\tikzstyle{mybox} = [draw=black, fill=white, very thick,
    rectangle, rounded corners, inner sep=10pt, inner ysep=10pt]
\tikzstyle{fancytitle} =[fill=black, text=white, font=\bfseries]

% Mihnea
\tikzstyle{mybox_code} = [mybox, draw = orange, fill=sandybrown]
\tikzstyle{fancytitle_code} = [fancytitle, fill = orange]

\definecolor{almond}{rgb}{0.94, 0.87, 0.8}
\definecolor{apricot}{rgb}{0.98, 0.81, 0.69}
\definecolor{atomictangerine}{rgb}{1.0, 0.6, 0.4}
\definecolor{sandybrown}{rgb}{0.96, 0.64, 0.38}
\definecolor{buff}{rgb}{0.94, 0.86, 0.51}

\definecolor{persianred}{rgb}{0.8, 0.2, 0.2}
\definecolor{papayawhip}{rgb}{1.0, 0.94, 0.84}
\tikzstyle{mybox_persianred} = [mybox, draw = persianred, fill=papayawhip]
\tikzstyle{fancytitle_persianred} = [fancytitle, fill = persianred]

\definecolor{whitesmoke}{rgb}{0.96, 0.96, 0.96}
\definecolor{wenge}{rgb}{0.39, 0.33, 0.32}
\tikzstyle{mybox_blue} = [mybox, draw = wenge, fill=whitesmoke]
\tikzstyle{fancytitle_blue} = [fancytitle, fill = wenge]

\definecolor{cerise}{rgb}{0.87, 0.19, 0.39}
\definecolor{mistyrose}{rgb}{1.0, 0.89, 0.88}
\tikzstyle{mybox_cerise} = [mybox, draw = cerise, fill=mistyrose]
\tikzstyle{fancytitle_cerise} = [fancytitle, fill = cerise]

\definecolor{pinegreen}{rgb}{0.0, 0.47, 0.44}
\definecolor{bubbles}{rgb}{0.91, 1.0, 1.0}
\tikzstyle{mybox_pinegreen} = [mybox, draw = pinegreen, fill=bubbles]
\tikzstyle{fancytitle_pinegreen} = [fancytitle, fill = pinegreen]

\definecolor{cream}{rgb}{1.0, 0.99, 0.82}
\definecolor{mikadoyellow}{rgb}{1.0, 0.77, 0.05}
\tikzstyle{mybox_mikadoyellow} = [mybox, draw = mikadoyellow, fill=cream]
\tikzstyle{fancytitle_mikadoyellow} = [fancytitle, fill = mikadoyellow]

\definecolor{cornsilk}{rgb}{1.0, 0.97, 0.86}
\tikzstyle{mybox_orange} = [mybox, draw = orange, fill=cornsilk]
\tikzstyle{fancytitle_orange} = [fancytitle, fill = orange]

\definecolor{aliceblue}{rgb}{0.94, 0.97, 1.0}
\definecolor{seagreen}{rgb}{0.18, 0.55, 0.34}
\tikzstyle{mybox_seagreen} = [mybox, draw = seagreen, fill=aliceblue]
\tikzstyle{fancytitle_seagreen} = [fancytitle, fill = seagreen]

\definecolor{jazzberryjam}{rgb}{0.65, 0.04, 0.37}
\definecolor{almond}{rgb}{0.94, 0.87, 0.8}
\tikzstyle{mybox_jazzberryjam} = [mybox, draw = jazzberryjam, fill=almond]
\tikzstyle{fancytitle_jazzberryjam} = [fancytitle, fill = jazzberryjam]

\definecolor{amaranth}{rgb}{0.9, 0.17, 0.31}
\definecolor{bisque}{rgb}{1.0, 0.89, 0.77}
\tikzstyle{mybox_amaranth} = [mybox, draw = amaranth, fill=bisque]
\tikzstyle{fancytitle_amaranth} = [fancytitle, fill = amaranth]

\definecolor{carminered}{rgb}{1.0, 0.0, 0.22}
\definecolor{blanchedalmond}{rgb}{1.0, 0.92, 0.8}
\tikzstyle{mybox_carminered} = [mybox, draw = amaranth, fill=blanchedalmond]
\tikzstyle{fancytitle_carminered} = [fancytitle, fill = carminered]

\definecolor{midnightgreen}{rgb}{0.0, 0.29, 0.33}
\definecolor{lavendermist}{rgb}{0.9, 0.9, 0.98}
\tikzstyle{mybox_midnightgreen} = [mybox, draw = midnightgreen, fill=lavendermist]
\tikzstyle{fancytitle_midnightgreen} = [fancytitle, fill = midnightgreen]

\definecolor{indigo}{rgb}{0.29, 0.0, 0.51}
\definecolor{isabelline}{rgb}{0.96, 0.94, 0.93}
\tikzstyle{mybox_indigo} = [mybox, draw = indigo, fill=isabelline]
\tikzstyle{fancytitle_indigo} = [fancytitle, fill = indigo]

\definecolor{russet}{rgb}{0.5, 0.27, 0.11}
\definecolor{ivory}{rgb}{1.0, 1.0, 0.94}
\tikzstyle{mybox_russet} = [mybox, draw = russet, fill=ivory]
\tikzstyle{fancytitle_russet} = [fancytitle, fill = russet]

\definecolor{neongreen}{rgb}{0.22, 0.88, 0.08}
\definecolor{splashedwhite}{rgb}{1.0, 0.99, 1.0}
\tikzstyle{mybox_neongreen} = [mybox, draw = neongreen, fill=splashedwhite]
\tikzstyle{fancytitle_neongreen} = [fancytitle, fill = neongreen]
%---------------------------------------------------------------------------------

\begin{tikzpicture}
\node [mybox_persianred] (box){%
    \begin{minipage}{0.3\textwidth}
    	{\centering\bf\color{persianred} (nume\_functie arg1 arg2 ...) \\}
		\begin{lstlisting}[language=Racket]
(max 2 3)               3
(+ 2 3)                 5
        \end{lstlisting}
    \end{minipage}
};

\node[fancytitle_persianred, right=10pt] at (box.north west) {Sintaxa Racket};
\end{tikzpicture}

%---------------------------------------------------------------------------------

\begin{tikzpicture}
\node [mybox_persianred] (box){%
    \begin{minipage}{0.3\textwidth}
%     	{\centering\bf\color{persianred} (nume\_functie arg1 arg2 ...) \\}
Valori booleene: \texttt{\#t \#f} (sau  \texttt{true false}) \\
Numere: \texttt{1 2 3 3.14 ...} \\
Simboli (literali): \texttt{'a 'b 'abc 'non-alnum?!}
	\end{minipage}
};

\node[fancytitle_persianred, right=10pt] at (box.north west) {Tipuri de bază};
\end{tikzpicture}


\begin{tikzpicture}
\node [mybox_seagreen] (box){%
    \begin{minipage}{0.3\textwidth}
    	{\centering\bf\small\color{seagreen} + - * / modulo quotient add1 sub1 \\}
		\begin{lstlisting}[language=Racket]
(+ 1 2)              3
(- 7 2)              5
(* 2 11)             22
(/ 5 2)              2.5
(quotient 5 2)       2
(modulo 5 2)         1
(add1 4)             5
(sub1 4)             3
        \end{lstlisting}
    \end{minipage}
};

\node[fancytitle_seagreen, right=10pt] at (box.north west) {Operatori aritmetici};
\end{tikzpicture}

%---------------------------------------------------------------------------------

\begin{tikzpicture}
\node [mybox_jazzberryjam] (box){%
    \begin{minipage}{0.3\textwidth}
    
{\centering\bf\small\color{jazzberryjam}  $<$ $<=$ $>$ $>=$ = eq? equal? zero? \\}
		\begin{lstlisting}[language=Racket]        
(< 3 2)                    #f
(>= 3 2)                   #t
(= 1 1)                    #t        (numere)
(= '(1 2) '(1 2))          eroare
(equal? '(1 2) '(1 2))     #t        (valori)
(eq? '(1 2 3) '(1 2 3))    #f        (referinte)

(define x '(1 2 3))
(eq? x x)                  #t

(zero? 0)                  #t (true)
(zero? 1)                  #f (false)
        \end{lstlisting}
    \end{minipage}
};

\node[fancytitle_jazzberryjam, right=10pt] at (box.north west) {Operatori relationali};
\end{tikzpicture}

%---------------------------------------------------------------------------------

\begin{tikzpicture}
\node [mybox_amaranth] (box){%
    \begin{minipage}{0.3\textwidth}
{\centering\bf\small\color{amaranth}  not and or \\}
		\begin{lstlisting}[language=Racket]
(not true)                      #f
(not false)                     #t
(or true false)                 #t
(and #f #t)                     #f
        \end{lstlisting}
    \end{minipage}
};

\node[fancytitle_amaranth, right=10pt] at (box.north west) {Operatori logici};
\end{tikzpicture}

%---------------------------------------------------------------------------------

% \begin{tikzpicture}
% \node [mybox_blue] (box){%
%     \begin{minipage}{0.3\textwidth}
% 		\begin{lstlisting}[language=Racket]
% '()
% null
%         \end{lstlisting}
%     \end{minipage}
% };

% \node[fancytitle_blue, right=10pt] at (box.north west) {Lista vida};
% \end{tikzpicture}

%---------------------------------------------------------------------------------

\begin{tikzpicture}
\node [mybox_cerise] (box){%
    \begin{minipage}{0.3\textwidth}
    	{\centering\bf\small\color{cerise} null '() cons list \\}
		\begin{lstlisting}[language=Racket]
'()									  ()
null								  ()

(cons 1 null)                         (1)
(cons 'a '(b c))                      (a b c)

(list 1)                              (1)
(list 1 2 3 4)                        (1 2 3 4)
        \end{lstlisting}
    \end{minipage}
};

\node[fancytitle_cerise, right=10pt] at (box.north west) {Constructori liste};
\end{tikzpicture}

%---------------------------------------------------------------------------------


\begin{tikzpicture}
\node [mybox_pinegreen] (box){%
    \begin{minipage}{0.3\textwidth}
    	{\centering\bf\small\color{pinegreen} car cdr first rest null? length member reverse append\\}
		\begin{lstlisting}[language=Racket]
(car '(1 2 3 4))                      1
(first '(1 2 3 4))                    1
(cdr '(1 2 3 4))                      (2 3 4)
(rest '(1 2 3 4))                     (2 3 4)
(cadr '(1 2 3 4 5))                   2
(cdar '(1 2 3 4 5))                   eroare
(cddr '(1 2 3 4 5))                   (3 4 5)
(caddr '(1 2 3 4 5))                  3

(null? null)                          #t (true)
(null? '(1 2))                        #f (false)

(length '(1 2 3 4))                   4
(length '(1 (2 3) 4))                 3

(member 'a '(b c a d a e))			  '(a d a e)
(member 'f '(b c a d a e))            #f

(reverse '(1 (2 3) 4))                (4 (2 3) 1)

(list? '())                           #t
(list? 2)                             #f

(append '(1 2 3) '(4) '(5))         (1 2 3 4 5)
(append 1 '(2 3 4))                 eroare
		\end{lstlisting}
    \end{minipage}
};

\node[fancytitle_pinegreen, right=10pt] at (box.north west) {Operatori pe liste};
\end{tikzpicture}



%---------------------------------------------------------------------------------

% \begin{tikzpicture}
% \node [mybox_mikadoyellow] (box){%
%     \begin{minipage}{0.3\textwidth}
% 		\begin{lstlisting}[language=Racket]
%         \end{lstlisting}
%     \end{minipage}
% };

% \node[fancytitle_mikadoyellow, right=10pt] at (box.north west) {Concatenarea listelor};
% \end{tikzpicture}

%---------------------------------------------------------------------------------

\begin{tikzpicture}
\node [mybox_orange] (box){%
    \begin{minipage}{0.3\textwidth}
		\begin{lstlisting}[language=Racket]
(take '(1 2 3 4) 2)              '(1 2)
(drop  '(1 2 3 4 5) 2)           '(3 4 5)

(take-right '(1 2 3 4) 2)        '(3 4)
(drop-right  '(1 2 3 4 5) 2)     '(1 2 3)
        \end{lstlisting}
    \end{minipage}
};

\node[fancytitle_orange, right=10pt] at (box.north west) {take și drop};
\end{tikzpicture}

%---------------------------------------------------------------------------------



% \begin{tikzpicture}
% \node [mybox_carminered] (box){%
%     \begin{minipage}{0.3\textwidth}
% 		\begin{lstlisting}[language=Racket]
% (define x 2)
% (define y (+ x 2))
% (+ x 1)                3
% (= y 4)                #t

% (define my_list '(a 2 3)) 
% (car my_list)           a
%         \end{lstlisting}
%     \end{minipage}
% };

% \node[fancytitle_carminered, right=10pt] at (box.north west) {Legarea variabilelor};
% \end{tikzpicture}

%---------------------------------------------------------------------------------

\vskip10ex % spațiere ca să nu intre în a doua coloană

\begin{tikzpicture}
\node [mybox_midnightgreen] (box){%
    \begin{minipage}{0.3\textwidth}
{\centering\bf\small\color{midnightgreen}    (lambda (arg1 arg2 ...) rezultat)  (define nume val)
% \[ CTRL\hspace{3pt}\backslash \hspace{10pt} \lambda \] \\
}
		\begin{lstlisting}[language=Racket]          
(lambda (x) x)                 functia identitate
((lambda (x) x) 2)          2  aplicare functie
(define idt (lambda (x) x))    legare la un nume
(define (idt x) x)             sintaxa alternativa
(idt 3)                     3
        \end{lstlisting}
        
% (lambda (l1 l2) (append l2 l1))
% (define append2 (lambda (l1 l2) (append l2 l1)))
% (define (append2 l1 l2) (append l2 l1))
% (append2 '(1 2 3) '(4 5 6))        (4 5 6 1 2 3)
    \end{minipage}
};

\node[fancytitle_midnightgreen, right=10pt] at (box.north west) {Funcții anonime (lambda) și funcții cu nume};
\end{tikzpicture}

%---------------------------------------------------------------------------------

\begin{tikzpicture}
\node [mybox_indigo] (box){%
    \begin{minipage}{0.3\textwidth}
		\begin{lstlisting}[language=Racket]
(if test exp1 exp2)

(if (< a 0)
    a 
    (if (> a 10) (* a a) 0))
        \end{lstlisting}
    \end{minipage}
};

\node[fancytitle_indigo, right=10pt] at (box.north west) {Sintaxa if};
\end{tikzpicture}

%---------------------------------------------------------------------------------

\begin{tikzpicture}
\node [mybox_russet] (box){%
    \begin{minipage}{0.3\textwidth}
		\begin{lstlisting}[language=Racket]
(cond  (test1 exp1) (test2 exp2) ... (else exp))

(cond
   ((< a 0) a) 
   ((> a 10) (* a a)) 
   (else 0))
        \end{lstlisting}
    \end{minipage}
};

\node[fancytitle_russet, right=10pt] at (box.north west) {Sintaxa cond};
\end{tikzpicture}

%---------------------------------------------------------------------------------

\begin{tikzpicture}
\node [mybox_persianred] (box){%
    \begin{minipage}{0.3\textwidth}\centering
%     	{\centering\bf\color{persianred}
		\begin{lstlisting}[language=Racket]
DA: (cons x L)        NU: (append (list x) L)
	                  NU: (append (cons x '()) L)
DA: (if c vt vf)      NU: (if (equal? c #t) vt vf)
DA: (null? L)         NU: (= (length L) 0)
DA: (zero? x)         NU: (equal? x 0)
DA: test              NU: (if test #t #f)
DA: (or ceva1 ceva2)  NU: (if ceva1 #t ceva2)
DA: (and ceva1 ceva2) NU: (if ceva1 ceva2 #f)
        \end{lstlisting}
    \end{minipage}
};

\node[fancytitle_persianred, right=10pt] at (box.north west) {AȘA  DA / AȘA NU};
\end{tikzpicture}


%---------------------------------------------------------------------------------

% \begin{tikzpicture}
% \node [mybox_code] (box){%
%     \begin{minipage}{0.3\textwidth}
% 		\begin{lstlisting}[language=Racket]

%         \end{lstlisting}
%     \end{minipage}
% };

% \node[fancytitle_code, right=10pt] at (box.north west) {Cum gandim un program};
% \end{tikzpicture}

%---------------------------------------------------------------------------------

\begin{tikzpicture}
\node [mybox_orange] (box){%
    \begin{minipage}{0.3\textwidth}\small
      \begin{enumerate}\itemsep.1ex
      	\item După ce variabilă(e) fac recursivitatea? (ce variabilă(e) se schimbă de la un apel la altul?)
        \item Care sunt condițiile de oprire în funcție de aceste variabile?(cazurile ``de bază'')
        \item Ce se întâmplă când problema nu este încă elementară? (Obligatoriu aici cel puțin un apel recursiv)
      \end{enumerate}
    \end{minipage}
};
\node[fancytitle_orange, right=10pt] at (box.north west) {Programare cu funcții recursive};
\end{tikzpicture}


%---------------------------------------------------------------------------------

\begin{tikzpicture}
\node [mybox_neongreen] (box){%
    \begin{minipage}{0.3\textwidth}\centering
\href{http://docs.racket-lang.org/}{http://docs.racket-lang.org/}
    \end{minipage}
};

\node[fancytitle_neongreen, right=10pt] at (box.north west) {Folosiți cu încredere!};
\end{tikzpicture}

%---------------------------------------------------------------------------------

\end{multicols*}
\end{document}

